%-------------------------------------------------------%
\documentclass[10pt]{article}
    \usepackage[dvipdfmx]{graphicx}
    \usepackage{bm,latexsym,amsmath,amssymb,amsfonts,mathrsfs}
    %-------------------------------------------------------%
    \usepackage{color}
    \input{colordvi.tex}
    %-------------------------------------------------------%
    \newcommand*{\D}{{\rm d}}
    %-------------------------------------------------------%
    \topmargin=0.0in
    \headsep=0.0in
    \headheight=0.0in
    \oddsidemargin=-0.22in
    \evensidemargin=-0.22in
    \textwidth=6.5in
    \textheight=9.0in
    %-------------------------------------------------------%
    \title{{\large {\bf 中間発表会概要集原稿の書き方}}}
    \author{}
    \date{}
    %
    \pagestyle{empty}
    %
    \begin{document}
    \maketitle
    %
    \thispagestyle{empty}
    %
    \vspace{-15mm}	
    \begin{flushright}
    {\bf 13CBnnnX~~アルバート~アインシュタイン}
    \\
    {\bf 13CBnnnS~~スティーブン~ワインバーグ}
    \\
    {\bf 指導教員~~湯川学}
    \end{flushright}
    
    
    \section{{\normalsize はじめに}}
     中間発表概要集の原稿の例です。
    できるだけこの\LaTeX{}のフォーマットに合わせてください。
    このファイルに直接書き込んでもよいですが、そうでない場合は下記をよく読み、
    できるだけ同じようにしてください。
    
    \section{{\normalsize 書式について}}
    \begin{itemize}
    \item 文書のマージン(余白)は上下左右とも20 mm。図も含めてこの範囲をはみ出さないように注意。
    \item タイトルのフォントは12ポイント、ゴシック体(MSゴシック、ヒラギノ角ゴなど)、中央揃え(センタリング)。
    \item 学生番号・氏名・指導教員のフォントは10ポイント、ゴシック体(MSゴシック、ヒラギノ角ゴなど)。
    \item 本文フォントは10ポイント、明朝体(MS明朝、ヒラギノ明朝など)。
    \item 本文中の見出しのフォントはゴシック体。
    \item カラーの図を入れても構わないが、白黒印刷をした際に明瞭に読めることを各自確認のこと。高解像度の図を含む場合は、適宜解像度を落として常識的なファイルサイズに収めること。
    \item 2ページ以内。ページ数は入れないこと
    \item \Red{PDFファイルに変換し、フォントを埋め込んだPDFファイルを提出すること。}
    \end{itemize}
    
    \section{{\normalsize 提出締切・方法}}
    \begin{itemize}
    \item 提出締切:9月29日(木) 17:00 \\
    (締切後の提出はできませんので注意してください)
    \item 提出方法:Blackboardに限る。\\
    Blackboardに「物理学科卒業研究2016」(「2016他-卒業研究: Thesis」と混同しないこと)という科目があるので、
    その科目の「教材/課題/テスト」$\to$「中間発表概要」から提出する。
    \end{itemize}
    
    
    \section{{\normalsize その他}}
     中間発表では、何を目的とした研究を行なっているか、卒研発表までにどこまで進む予定なのか、
    現状ではどこまで進んでいるのか、ということを、他の研究室の教員・学生(卒研生だけでなく3年生以下も含む)
    に説明する必要があります。指導教員とよく相談をして、十分に準備をして臨んでください。
    
    
    
    \end{document}
    %-------------------------------------------------------%